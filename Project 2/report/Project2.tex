\documentclass{article}

\usepackage{amsmath,amsfonts,amsthm,amssymb}
\usepackage{fancyhdr}
\usepackage{float}
\usepackage{lastpage}
\usepackage{graphicx}
\usepackage{supertabular}
\usepackage{multirow}
\usepackage{ifthen}
\usepackage{enumerate}
\usepackage[hidelinks]{hyperref}
\usepackage{soul}

% In case you need to adjust margins:
\topmargin=-0.5in
\evensidemargin=0in
\oddsidemargin=0in
\textwidth=6.5in
\textheight=9.0in
\headsep=0.25in

% Homework Specific Information
\newcommand{\hmwkTitle}{Project Report 2}
\newcommand{\hmwkDueDate}{March 17th, 2014}
\newcommand{\hmwkClass}{42-731}
\newcommand{\hmwkAuthor}{Alex Sun Yoo, Michael Nye, Ozan Iskilibli}
\newcommand{\hmwkEmail}{ayoo, mnye, oiskilib}
\newcommand{\hmwkCollaborators}{}
\newcommand{\bigspace}{\vspace{.25in}}

% Tools for formatting questions
\newcommand{\question}[1] {\vspace{.25in} \hrule\vspace{0.5em}
\noindent{\bf #1} \vspace{0.5em}
\hrule \vspace{.10in}}
\renewcommand{\part}[1] {\vspace{.10in} {\bf (#1)}}

% Setup the header and footer
\pagestyle{fancyplain}
\lhead{\fancyplain{}{\hmwkAuthor \\ \hmwkEmail}}
\chead{\fancyplain{}{\textbf{\hmwkTitle}}}
\rhead{\fancyplain{}{\hmwkClass \\ Due:\ \hmwkDueDate}}
\lfoot{}
\cfoot{}
\rfoot{Page\ \thepage\ of\ \pageref{LastPage}}
\renewcommand\headrulewidth{0.4pt}
\renewcommand\footrulewidth{0.4pt}

% Format paragraphs to have spacing instead of indents
\setlength{\parindent}{0pt}
\setlength{\parskip}{5pt plus 1pt}

% This is used to trace down (pin point) problems
% in latexing a document:
%\tracingall

\renewcommand{\arraystretch}{1.5}

%%%%%%%%%%%%%%%%%%%%%%%%%%%%%%%%%%%%%%%%%%%%%%%%%%%%%%%%%%%%%

\begin{document}

\thispagestyle{plain}
\begin{center}
{\Large \hmwkClass\ \hmwkTitle} \\
\hmwkAuthor \\
\hmwkEmail \\
\ifthenelse{\equal{\hmwkCollaborators}{}}{}{Collaborators: \hmwkCollaborators\\}
Due: \hmwkDueDate\\
\end{center}
%%%%%%%%%%%%%%%%%%%%%%%%%%%%%%%%%%%%%%%%%%%%%%%%%%%%%%%%%%%%%

\section*{Part 1: Particle Feature Detection}

\subsection*{B.2.1}

When running the script, a rectangle is manually selected by the user. It is assumed to contain only background, and simple statistics are calculated.
The standard deviation and mean of the background region is calculated and was used for calibration.

\subsection*{B.2.2}

To find all local extrema, the image was filtered with a Gaussian low pass filter, set to have a $\sigma$ of $\frac{1}{3}$ the Rayleigh limit. We then mark extrema as extrema only if they are the largest/smallest pixel inside their mask region.

Using a 5x5 mask detects noticeably fewer local extrema than using a 3x3 mask, as shown in Figures ~\ref{fig:extrema_3} and ~\ref{fig:extrema_5}. This is because with a larger mask, the tested point is compared with a larger area of neighboring points, reducing its chances of being classified as a local extrema.

The minima that are detected in one and not the other appear mostly in the background region of the image, and not in the strip of particles we are actually attempting to locate. This suggests the larger mask is better at rejecting false positives. However, the 5x5 mask seemed to reject some real particles as well. Our final results were generated using a 3x3 mask.

\begin{figure}[H]
\centering
\includegraphics[width=16cm]{figures/extrema_3_full.png}
\includegraphics[width=16cm]{figures/extrema_3_zoom.png}
\caption{Local Extrema (3x3 mask), full and zoomed}
\label{fig:extrema_3}
\end{figure}

\begin{figure}[H]
\centering
\includegraphics[width=16cm]{figures/extrema_5_full.png}
\includegraphics[width=16cm]{figures/extrema_5_zoom.png}
\caption{Local Extrema (5x5 mask), full and zoomed}
\label{fig:extrema_5}
\end{figure}



\subsection*{B.2.3}

Detected local maxima and minima in part B.2.2 are further processed by Delaunay Triangulation\cite{delaunaywiki}\cite{delaunaymathworks} as the next step. Delaunay Triangulation creates sets of triangles from given data points, whose circumcircles do not contain any vertices of neighboring triangles. In addition, this method aims to maximize the smallest angle in each created triangle. If the local minima are triangulated, then the local maxima could be assigned to each triangle, eliminating some potentially unnecessary local maxima. If a Delaunay Triangle of local minima contain more than 1 local maxima, only the local maxima that has the highest intensity is accepted and the rest are assumed to be artifacts. The method also assigns three distinct minima to each maxima, based on what triangle the maxima is contained by. These minima are used to compute how prominent the maxima are.

For this implementation, the MATLAB built-in function \texttt{DelaunayTri} was used. After that, the maxima that fall into each triangle are found. For this purpose, the matrix of local maxima and each minima triangle are swept with another built-in function, \texttt{inpolygon}. This function returns the maxima that are inside or on top of the triangle vertices. Finally, among the found local maxima, the one with the highest intensity is stored as the \emph{associated local maxima} for each triangle. The triangles and their associated local maxima are shown in Figure \ref{fig:assocLocal}.

\begin{figure}[H]
\centering
\includegraphics[width=16cm]{figures/delaunay_full.png}
\includegraphics[width=16cm]{figures/delaunay_zoom.png}
\caption{Associated local minima and maxima, where the minima are located at Delaunay Triangle corners and maxima are tagged with red crosses, full and zoomed}
\label{fig:assocLocal}
\end{figure}



\subsection*{B.2.4}
 
The associated maxima found by Delaunay Triangulation are further investigated in this part, in order to quantify a level of confidence for the detected particles. The question allowed to remove the speckle intensity variation term ($\sigma^2(I_L)$) in equation (4) of reference \cite{ponti}, as well as assuming that the local minima intensities' mean and variance match the cropped region's intensity mean and variance in Part B.2.1. Thanks to these simplifications, the task here is reduced to a comparison of the background to the detected speckle in terms of intensity, with a chosen level of confidence. If this task is expressed in mathematical form, following \cite{ponti}:

\begin{align}
I_{maxima} - \mu_{BG} &= Q\cdot \sigma_{\Delta I}\\
\intertext{where $I_{maxima}$ is the intensity of the local maxima, $\mu_{BG}$ is the mean intensity of the background, $Q$ is the selection quantile for the \emph{t-test}, and the $\sigma_{\Delta I}$ is defined as}
\sigma^2_{\Delta I} &= \sigma^2(I_L) + \frac{1}{3}\sigma^2(I_{BG}) \nonumber \\
\intertext{For $\sigma^2(I_L)=0$ is given, this simplifies to}
\sigma^2_{\Delta I} &= \frac{1}{3}\sigma^2(I_{BG})
\end{align}

The only variable that is not defined by the image is the selection quantile, Q. It is, however, recommended to be chosen as a value, such that it matches the speckles that could be seen by bare eyes, when the images are shown in MATLAB. For this reason, the Q is chosen as 10. The resulting speckle detection is shown in figure \ref{fig:statMax}.

\begin{figure}[H]
\centering
\includegraphics[width=16cm]{figures/statistics_maxima_full.png}
\includegraphics[width=16cm]{figures/statistics_maxima_zoom.png}
\caption{The statistically chosen maxima, tagged over the image with red crosses,
    full and zoomed}
\label{fig:statMax}
\end{figure}
 


\pagebreak
\section*{Part 2: Sub-pixel Resolution Particle Detection}

\subsection*{B.3.1}

Synthetic image generation is largely similar to the process used by Cheezum et al.\cite{cheezum}. First, a list of maxima are used to generate single pixel impulses. The intensity of these impulses is set to the intensity of the original image at the locations of the maxima. The image is then filtered using a Gaussian kernel to simulate the main lobe of our point spread function, and Gaussian white noise is added on top of the image.

The levels of our white noise was chosen based on the computed background statistics from part B.2.1. The results of this process can be seen in Figure \ref{fig:syntheticImage}; it closely resembles the actual image it was generated from.

\begin{figure}[H]
\centering
\includegraphics[width=16cm]{figures/synthetic_image.png}
\caption{A sample synthetic image, and the original image it was generated from}
\label{fig:syntheticImage}
\end{figure}


\subsection*{B.3.2}

To perform subpixel particle detection, the test image is first upsampled by a factor of 5 using interpolation. The MATLAB built-in command \texttt{interp2} was used for linear interpolation. After this, a Gaussian kernel is created with the side of the Rayleigh limit. Then, starting at each pixel-resolution maxima found in Part 1, pixels within 5 pixels of the maxima was tested. The Gaussian kernel is scaled by the center point at each location, and the error between the image and the kernel is calculated by subtracting the two, taking the absolute value, then summing up all of the values in the resulting matrix. The subpixel with the lowest error is chosen as the true particle location for each maxima. Figure \ref{fig:subpixelReal} has an example of sample subpixel detection results.

\begin{figure}[H]
\centering
\includegraphics[width=16cm]{figures/subpixel_real_full.png}
\includegraphics[width=16cm]{figures/subpixel_real_zoom.png}
\caption{The detected particle locations using subpixel Gaussian fitting, full and zoomed}
\label{fig:subpixelReal}
\end{figure}


\subsection*{B.3.3}

To test the performance of the subpixel Gaussian fitting algorithm, we identify the locations of our particles in the synthetic image. Since we know exactly where these particles were located, we compared the results returned by the Gaussian fitting with the known center coordinates. The error between the detected and actual x- and y- coordinates was calculated, and found to have the following statistics:

\begin{center}
\begin{tabular}{r | c | c}
               & y-coordinate & x-coordinate \\ \hline
    Mean Error & -0.042 px      & 0 px       \\
    Error STD  & 0.202 px       & 0.206 px
\end{tabular}
\end{center}

The fitting was found to be largely unbiased, and accurate to within a fifth of a pixel. It correctly reported the location of 44 out of our 48 particles. Note that the test synthetic image had low noise; its background noise followed the same statistics as the (relatively clean) source image it was generated from. This suggests the performance of the sub-pixel fitting is strong under high quality imaging.

\begin{figure}[H]
\centering
\includegraphics[width=16cm]{figures/subpixel_synthetic_full.png}
\includegraphics[width=16cm]{figures/subpixel_synthetic_zoom.png}
\caption{The detected particle locations using subpixel Gaussian fitting on a
    synthetic image, full and zoomed}
\label{fig:subpixelSynthetic}
\end{figure}



\pagebreak
\section*{Appendix A: Reproducing The Results}
 
The results could be reproduced as in the class presentation and this report, by setting 2 parameters in the main project m-file, ``pr2.m'', the following way.
\begin{enumerate}
\item For processing single image, set
\begin{verbatim}
Nimages = 1;
\end{verbatim}
If the figures are desired to be popping out in MATLAB, set boolean variable `plotting' from `false' to
\begin{verbatim}
plotting = true;
\end{verbatim}
\item For batch processing the image set, recommended setting is
\begin{verbatim}
Nimages = numel(images);
plotting = false;
\end{verbatim}
The MATLAB environment files in `.mat' format are stored in the `mat\_files' directory in the project root.
\end{enumerate}
 


\pagebreak
\begin{thebibliography}{9}
\fontsize{10pt}{12pt}\selectfont
\raggedright

    \bibitem{ponti}
        A. Ponti, P. Vallotton, W. C. Salmon, C. M. Waterman-Storer, and G.
        Danuser, \ul{Computational analysis of F-actin turnover in cortical
        actin meshworks using fluorescent speckle microscopy}, {\em Biophysical
        Journal}, 84:3336-3352, 2003.

    \bibitem{cheezum}
        M. K. Cheezum, W. F. Walker, and W. H. Guilford, 
        \ul{Quantitative comparison of algorithms for tracking single 
        fluororescent particles}, {\em Biophysical Journal}, 81:2378- 2388, 2001.

    \bibitem{delaunaywiki}
        Wikipedia -- Delaunay triangulation
        \url{http://en.wikipedia.org/wiki/Delaunay_triangulation}
    
    \bibitem{delaunaymathworks}
        Mathworks -- Delaunay triangulation
        \url{http://www.mathworks.com/help/matlab/ref/delaunay.html}

\end{thebibliography}


%%%%%%%%%%%%%%%%%%%%%%%%%%%%%%%%%%%%%%%%%%%%%%%%%%%%%%%%%%%%%

\end{document}

%%%%%%%%%%%%%%%%%%%%%%%%%%%%%%%%%%%%%%%%%%%%%%%%%%%%%%%%%%%%%
